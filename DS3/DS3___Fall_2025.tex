\documentclass[9pt, handout]{beamer}
\usetheme{CambridgeUS}
\usepackage{xcolor}
\usepackage{geometry}
\usepackage{array}
\usepackage{comment}

\AtBeginSection[]
{
  \begin{frame}
    \frametitle{Table of Contents}
    \tableofcontents[currentsection]
  \end{frame}
}

\setbeamertemplate{footline}
{
  \leavevmode%
  \hbox{%
    \begin{beamercolorbox}[wd=.333333\paperwidth,ht=2.25ex,dp=1ex,center]{author in head/foot}%
      \usebeamerfont{author in head/foot}\insertshortauthor
    \end{beamercolorbox}%
    \begin{beamercolorbox}[wd=.333333\paperwidth,ht=2.25ex,dp=1ex,center]{title in head/foot}%
      \usebeamerfont{title in head/foot}\insertshortsubtitle
    \end{beamercolorbox}%
    \begin{beamercolorbox}[wd=.333333\paperwidth,ht=2.25ex,dp=1ex,right]{date in head/foot}%
      \usebeamerfont{date in head/foot}\insertshortdate{}\hspace*{2em}
      \usebeamertemplate{page number in head/foot}\hspace*{2ex}
    \end{beamercolorbox}
  }%
  \vskip0pt%
}

\title{Principles of Economics}
\subtitle{Discussion Session 3: Government Intervention}
\author{Joe Wilske and Yuzhi Yao}
\institute{Boston College}
\date{\today}

\begin{document}

\frame{\titlepage}

\begin{frame}{Price Controls}
    \begin{itemize}
        \item \textbf{Price ceiling:} illegal to sell above the specified price.
        \begin{itemize}
            \item Rent controls, anti price-gouging laws
            \item Creates a gap in quantity demanded and quantity supplied (shortage): $\: Q^D > Q^S$.
            \vspace{5pt}
        \end{itemize}
        \item \textbf{Price floor:} illegal to sell below the specified price.
        \begin{itemize}
            \item Minimum wage
            \item Also creates a gap in quantity demanded and supplied (surplus): $\: Q^D < Q^S$.
        \end{itemize}
        \vspace{5pt}
        \item If the original equilibrium is already in compliance with the price control, then we say the price control is \textbf{non-binding} -- it has no effect.
    \end{itemize}
\end{frame}

\begin{frame}{Exercise 1: Price Controls}
    Consider the orange juice market: 
    \begin{itemize}
        \item[-] Demand: $Q^D = 50 - P$
        \item[-] Supply: $Q^S = 4P$
    \end{itemize}
    \begin{enumerate}
        \item Without government intervention, what are the equilibrium price and quantity? 
        \item Suppose now the government imposes rent controls (a price ceiling) of \$8 to this market. Will there be a shortage or surplus? Calculate the shortage/surplus if it exists. 
        \item Suppose the price ceiling is \$12. Will there be a shortage or surplus? Calculate the shortage/surplus if it exists. 
    \end{enumerate}
    \vspace{1in}
\end{frame}

\begin{frame}{Exercise 1: Price Controls}
    Solution: 
    \begin{enumerate}
        \item In equilibrium, quantity demanded equals quantity supplied at the equilibrium price: $Q^D(P)=Q^S(P)$. This gives us: 
        \begin{align*}
            & 50 - P = 4P \\
            & Q = 40, P = 10
        \end{align*}
        \item Under the price ceiling, $Q^D = 50 - 8 =42$, $Q^S=4\times 8 = 32$. Since the quantity demanded is larger than the quantity supplied, we have a shortage of $42-32=10$. 
        \item No shortage or surplus since the price ceiling is not binding. 
    \end{enumerate}
\end{frame}

\begin{frame}{Taxes}
    \begin{itemize}
        \item Creates a gap between the price paid by the buyer $P^D$, and the price received by the seller $P^S$.
        \item Suppose the government imposes a tax of size $t$ on buyers.\\
        $\implies \text{Condition 1:} \quad P^D = P^S + t$
        \item To find new equilibrium, we set
        \begin{align*}
            Q^D(P^D) &= Q^S(P^S)\\
            Q^D(P^S + t) &= Q^S(P^S)
        \end{align*}
        \item Suppose the government instead imposes a tax of size $t$ on sellers.\\
        $\implies \text{Condition 2:} \quad P^S = P^D - t$
        \item But this is equivalent to Condition 1!  So we can find the new equilibrium using the same setup:
        \begin{align*}
            Q^D(P^D) &= Q^S(P^S)\\
            Q^D(P^S + t) &= Q^S(P^S)
        \end{align*}
    \end{itemize}
    \textbf{Punchline:} The resulting equilibrium prices and quantity are the same whether the tax is imposed on buyers or sellers.
\end{frame}

\begin{frame}{Exercise 2: Tax}
Consider the yogurt market: 
\begin{itemize}
    \item[-] Demand: $Q^D = 400 - 10P$
    \item[-] Supply: $Q^S = 30P - 800$
\end{itemize}
\begin{enumerate}
    \item Without government intervention, what are the equilibrium price and quantity?
    \item Suppose now the government imposes a tax of \$4 per unit to the buyers. What are the new equilibrium prices and quantity?
    \item How is the tax split between buyers and sellers? What role does elasticity play in this?
\end{enumerate}
\vspace{1in}
\end{frame}

\begin{frame}{Exercise 2: Tax}
    Solution: 
    \begin{enumerate}
        \item In equilibrium, quantity demanded equals quantity supplied at the equilibrium price: $Q^D(P)=Q^S(P)$. This gives us: 
        \begin{align*}
            & 400 - 10P = 30P - 800 \\
            & Q = 100, P = 30
        \end{align*}
        \item The price paid by buyers is $P^D = P^S + 4$, and the price received by sellers is $P^S$. In equilibrium, quantity demanded at the price paid by buyers equals to quantity supplied at the price received by sellers. This gives us: 
        \begin{align*}
            Q^D(P^D) &= Q^S(P^S)\\
            Q^D(P^S + 4) &= Q^S(P^S)\\
            400 - 10(P^S + 4) &= 30P^S - 800 \\
            P^S = 29,\:P^D &= 33, \: Q = 70
        \end{align*}
        \item Compared to the market without tax, the price paid by buyers increases from \$30 to \$33, so the tax shared by buyers is \$3. Similarly, the price received by sellers decreases from \$30 to \$29, and the tax shared by sellers is \$1.  The agent with the more elastic (flatter) curve will pay less of the tax.
    \end{enumerate}
\end{frame}

\begin{frame}{Subsidies}
    \begin{itemize}
        \item The opposite of a tax: for every unit sold, the government \textit{contributes} money rather than takes it.
        \vspace{5pt}
        \item Consider a subsidy to buyers of size $\sigma$.\\
        $\implies \text{Condition 3:} \quad P^D = P^S - \sigma$
        \item Likewise, a subsidy to sellers of size $\sigma$.\\
        $\implies \text{Condition 4:} \quad P^S = P^D + \sigma$.
        \vspace{5pt}
        \item As before, the two conditions are equivalent, so \textbf{the resulting prices and quantity are the same either way}.
        \item To solve, set
        \begin{align*}
            Q^D(P^D) &= Q^S(P^S)\\
            Q^D(P^S - \sigma) &= Q^S(P^S)
        \end{align*}
    \end{itemize}
\end{frame}

\begin{frame}{Exercise 3: Subsidy}

    Consider the market for corn:
    \begin{itemize}
        \item $Q^D = 152 - 7P$
        \item $Q^S = P - 8$
    \end{itemize}
    \vspace{10pt}
    \begin{enumerate}
        \item What are the original equilibrium price and quantity?
        \item Suppose the government offers a subsidy of \$8 per unit to the sellers. What are the new equilibrium prices and quantity?
        \item How are the benefits of the subsidy split between buyers and sellers?  What role does elasticity play here?
    \end{enumerate}
\end{frame}

\begin{frame}{Exercise 3: Subsidy}
    Solution:
    \begin{enumerate}
        \item In equilibrium, 
        \begin{align*}
            Q^D(P) &= Q^S(P)\\
            152 - 7P &= P - 8\\
            P = 20&, \: Q = 12
        \end{align*}
        \item With a subsidy $\sigma = 8$ to sellers, we know $P^S = P^D + 8$. So set
        \begin{align*}
            Q^D(P^D) &= Q^S(P^S)\\
            Q^D(P^D) &= Q^S(P^D + 8)\\
            152 - 7P^D &= (P^D + 8) - 8\\
            P^D = 19, \: P^S &= 27, \: Q = 19
        \end{align*}
        \item Compared the the original equilibrium, consumers pay \$1 less, and sellers receive \$7 more.  The agent with the more elastic (flatter) curve will receive less of the subsidy.
    \end{enumerate}
\end{frame}

\end{document}
