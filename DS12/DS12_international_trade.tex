\documentclass[9pt]{beamer}
\usetheme{CambridgeUS}
\usepackage{xcolor}
\usepackage{geometry}
\usepackage{array}
\usepackage[export]{adjustbox}
\usepackage{comment}
\usepackage{caption}
\usepackage{subcaption}
\usepackage{siunitx}
\sisetup{group-separator = {,}, group-minimum-digits = 3}



\AtBeginSection[]
{
  \begin{frame}
    \frametitle{Table of Contents}
    \tableofcontents[currentsection]
  \end{frame}
}

\setbeamertemplate{footline}
{
  \leavevmode%
  \hbox{%
    \begin{beamercolorbox}[wd=.333333\paperwidth,ht=2.25ex,dp=1ex,center]{author in head/foot}%
      \usebeamerfont{author in head/foot}\insertshortauthor
    \end{beamercolorbox}%
    \begin{beamercolorbox}[wd=.333333\paperwidth,ht=2.25ex,dp=1ex,center]{title in head/foot}%
      \usebeamerfont{title in head/foot}\insertshortsubtitle
    \end{beamercolorbox}%
    \begin{beamercolorbox}[wd=.333333\paperwidth,ht=2.25ex,dp=1ex,right]{date in head/foot}%
      \usebeamerfont{date in head/foot}\insertshortdate{}\hspace*{2em}
      \usebeamertemplate{page number in head/foot}\hspace*{2ex}
    \end{beamercolorbox}
  }%
  \vskip0pt%
}

\title{Principles of Economics}
\subtitle{Discussion Session 12: International Trade}
\author{Joe Wilske}
\institute{Boston College}
\date{\today}

\begin{document}

\frame{\titlepage}

\begin{frame}{Why trade?}
\begin{itemize}
    \item \textbf{Autarky:} The state of total trade isolation.  No goods cross borders.
\end{itemize}
\vspace{10pt}
David Ricardo (1817):
    \begin{itemize}
        \item Suppose England and Portugal can both produce wine and cloth, but Portugal is more productive at producing both.
        \item[$\rightarrow$] What could Portugal gain from trading with England?
    \end{itemize}
\end{frame}

\begin{frame}{Comparative Advantage}
David Ricardo (1817):
\begin{itemize}
    \item Suppose this table gives the hours of labor required to produce one unit of wine and cloth in England and Portugal.
\end{itemize}
    \begin{table}[h!]
    \centering
    \begin{tabular}{lccc}
    \hline
    \textbf{Country} & \textbf{Cloth} & \textbf{Wine} \\
    \hline
    England  & 15 & 20 \\
    Portugal & 10 & 10 \\
    \hline
    \end{tabular}
    \end{table}
\begin{itemize}
    \item Portugal has an \textit{absolute advantage} in each:  producing both wine and cloth is less costly in Portugal than in England.
    \vspace{5pt}
    \pause
    \item But England has a \textit{comparative advantage} in cloth:  the opportunity cost of producing cloth is lower in England than in Portugal.
    \vspace{5pt}
    \begin{itemize}
        \item In Portugal, producing 1 unit of cloth means forgoing $\frac{10}{10} = 1$ unit of wine.
        \vspace{3pt}
        \item In England, producing 1 unit of cloth means forgoing $\frac{15}{20} = 0.75$ units of wine.
    \end{itemize}
    \vspace{3pt}
    \pause
    \item[$\implies$] The autarky price of cloth is lower in England than in Portugal.
    \vspace{3pt}
    \item[$\implies$] Both countries benefit from specializing in their comparative advantage and trading with each other
\end{itemize}
\end{frame}

\begin{frame}{Direction of Trade Flows}
    \begin{itemize}
        \item England has comparative advantage in cloth.
        \item Portugal has comparative advantage in wine.
        \vspace{10pt}
        \pause
        \item[$\implies$] In \textit{autarky}, the price of cloth is lower in England than in Portugal.
        \vspace{10pt}
        \pause
        \item[$\implies$] Under free trade, England will export cloth to Portugal and import wine from Portugal.
        \pause
        \vspace{15pt}
        \item In general, under free trade, \textbf{each country will export its comparative advantage and import its comparative disadvantage.}
    \end{itemize}
\end{frame}

\begin{frame}{Exercise 1: Comparative Advantage}
    \begin{itemize}
        \item The following table gives autarky equilibrium prices for automobiles in the US and Mexico (both in USD).
    \end{itemize}
    \vspace{5pt}
    \begin{table}[h!]
    \centering
    \begin{tabular}{lccc}
    \hline
    \textbf{Country} & \textbf{Automobile Price} \\
    \hline
    United States & 45,000 \\
    Mexico & 40,000 \\
    \hline
    \end{tabular}
    \end{table}
    \vspace{5pt}
    \begin{enumerate}
        \item Which country has the comparative advantage in automobiles?\\
        \vspace{5pt}
        \item Will the US export or import automobiles?
    \end{enumerate}
\end{frame}

\begin{frame}{Exercise 1: Comparative Advantage}
    Solution:
    \vspace{10pt}
    \begin{enumerate}
        \item $\num{40000} < \num{45000}$, so Mexico has the comparative advantage in automobile production.
        \vspace{5pt}
        \item The US will import automobiles.
    \end{enumerate}
\end{frame}

\begin{frame}{Gains from Trade}

    \only<1>{
    \begin{figure}
        \centering
        \includegraphics[width=1\linewidth]{Slide1.PNG}
        \caption*{The above figure shows the autarky equilibria for the English cloth and wine markets.}
    \end{figure}
    }
    \only<2>{
    \begin{figure}
        \centering
        \includegraphics[width=1\linewidth]{Slide2.PNG}
        \caption*{Autarky CS and PS are calculated as usual.}
    \end{figure}
    }
    \only<3>{
    \begin{figure}
        \centering
        \includegraphics[width=1\linewidth]{Slide3.PNG}
        \caption*{Now introduce the prices for ``Rest of World".}
    \end{figure}
    }
    \only<4>{
    \begin{figure}
        \centering
        \includegraphics[width=1\linewidth]{Slide4.PNG}
        \caption*{Suppose England abandons autarky for free trade. English buyers and sellers now use \textit{RoW} prices.}
    \end{figure}
    }
    \only<5>{
    \begin{figure}
        \centering
        \includegraphics[width=1\linewidth]{Slide5.PNG}
        \caption*{The differences between quantities supplied and demanded by English producers and consumers are made up by exports and imports.}
    \end{figure}
    }
    \only<6>{
    \begin{figure}
        \centering
        \includegraphics[width=1\linewidth]{Slide6.PNG}
        \caption*{\textbf{Balanced Trade:} The value of all exports equals the value of all imports.  $\implies$  If the only goods that England trades are cloth and wine, then the areas of the pink rectangles must be equal.}
    \end{figure}
    }
    \only<7>{
    \begin{figure}
        \centering
        \includegraphics[width=1\linewidth]{Slide7.PNG}
        \caption*{CS is the difference between the demand curve and the \textit{RoW} price.\\
        PS is the difference between the \textit{RoW} price and the supply curve.}
    \end{figure}
    }
    \only<8>{
    \begin{figure}
        \centering
        \includegraphics[width=1\linewidth]{Slide8.PNG}
        \caption*{To recap, this is CS and PS under autarky.}
    \end{figure}
    }
    \only<9>{
    \begin{figure}
        \centering
        \includegraphics[width=1\linewidth]{Slide7.PNG}
        \caption*{Free trade increases PS and reduces CS for the country's comparative advantage good.\\
        Free trade reduces PS and increases CS for the country's comparative disadvantage good.}
    \end{figure}
    }
    \only<10>{
    \begin{figure}
        \centering
        \includegraphics[width=1\linewidth]{Slide9.PNG}
        \caption*{The addition of the dark triangles guarantees that the gains to the winners are greater than the losses to the losers.}
    \end{figure}
    }
\end{frame}

\begin{frame}{Exercise 2: Gains from Trade}
Consider the market for automobiles in the US under autarky:
    \begin{itemize}
        \item $Q^S = P - \num{20000}$
        \item $Q^D = -\frac{1}{2}P + \num{40000}$
    \end{itemize}
    \vspace{10pt}
    \begin{enumerate}
        \item What are the autarky equilibrium price and quantity?
        \item What are CS and PS?
    \end{enumerate}
    \vspace{10pt}
    Now suppose the \textit{Rest of World} price is $\$\num{30000}$ and the US adopts free trade. \\ Assume for the sake of simplicity that the US is a `small country'.
    \vspace{10pt}
    \begin{enumerate}
        \setcounter{enumi}{2}
        \item Does the US have a comparative advantage in automobiles?
        \item What are the new quantities supplied and demanded?
        \item How many cars does the US import or export?
        \item What is the change in CS and PS from the autarky equilibrium??
        \item What is the change in total surplus?
    \end{enumerate}
\end{frame}

\begin{frame}{Exercise 2: Gains from Trade}
    Solution:
    \begin{enumerate}
        \item $P^{\text{A}} = \num{40000} \: \text{ and } \: Q^{\text{A}} = \num{20000}$.
        \item $CS^A = \frac{1}{2} \times \num{20000} \times \num{40000} = \num{400000000} \: \text{ and }$ \\ \vspace{1pt} $PS^A = \frac{1}{2} \times \num{20000} \times \num{20000} = \num{200000000}$.
        \item No.  The \textit{RoW} price is lower than the US autarky price.
        \item $Q^S = (\num{30000}) - \num{20000} = \num{10000} \: \text{ and } \: Q^D = -\frac{1}{2}(\num{30000}) + \num{40000} = \num{25000}$.
        \item The US imports $Q^D - Q^S = \num{15000}$ automobiles.
        \item $CS^{FT} = \frac{1}{2} \times \num{25000} \times \num{50000} = \num{625000000} \: \text{ and }$ \\ \vspace{1pt} $PS^{FT} = \frac{1}{2} \times \num{10000} \times \num{10000} = \num{50000000}$.\\
        \vspace{1pt} $\implies \:$ $CS$ increases by \$225 million and $PS$ decreases by \$150 million.
        \item Total surplus increases by \$75 million.
    \end{enumerate}
\end{frame}

\begin{frame}{Protectionism}
    \begin{itemize}
        \item We've seen that free trade creates winners and losers.
        \vspace{2pt}
        \begin{itemize}
            \item The losses to the losers are smaller than the gains to the winners.
            \vspace{3pt}
            \item Free trade is beneficial on net.
        \end{itemize}
        \vspace{4pt}
        \item If the losers are organized, they may lobby for \textbf{protectionism}: governmental policies that restrict trade in order to benefit certain groups.
        \vspace{4pt}
        \item The previous problem concerned automobile manufacturing:
        \vspace{2pt}
        \begin{itemize}
            \item Consumers benefit from free trade, but organization of US car-buyers is difficult.
            \vspace{3pt}
            \item American auto manufacturers and labor suffer from free trade; organization is easy.\\
            \vspace{3pt}
            $\implies \:$ Those organized groups lobby for protectionist policies.
        \end{itemize}
    \end{itemize}
\end{frame}

\begin{frame}{Tariffs}
    \begin{itemize}
        \item The most common protectionist policy is a \textbf{tariff}: a tax on imported goods.
        \vspace{3pt}
        \item The effect of a tariff is much the same as a tax: \\ 
        \vspace{1pt}
        it raises the price to the buyers and decreases the quantity sold.
        \vspace{3pt}
        \item Tariffs effectively push us back toward the autarky equilibrium, reversing both the harmful and beneficial effects of free trade.
    \end{itemize}
\end{frame}

\begin{frame}{Lobbying for Tariffs}
    \begin{figure}
        \centering
        \includegraphics[width=.59\linewidth]{uaw_lobbying.png}
        \caption{From UAW.org}
        \label{fig:placeholder}
    \end{figure}
\end{frame}

\begin{frame}{Tariffs Graphically}
    \only<1>{
    \begin{figure}
        \centering
        \includegraphics[width=1\linewidth]{Slide10.PNG}
        \caption*{Consider the autarky equilibrium for automobiles in the US, with \textit{Rest of World} price $P^{RoW}$.}
    \end{figure}
    }
    \only<2>{
    \begin{figure}
        \centering
        \includegraphics[width=1\linewidth]{Slide11.PNG}
        \caption*{Under free trade, Americans use $P^{RoW}$, so $CS$ expands and $PS$ contracts.}
    \end{figure}
    }
    \only<3>{
    \begin{figure}
        \centering
        \includegraphics[width=1\linewidth]{Slide12.PNG}
        \caption*{Suppose the government introduces a tariff, which raises the domestic price to $P^T \in (P^{RoW}, P^A)$.}
    \end{figure}
    }
    \only<4>{
    \begin{figure}
        \centering
        \includegraphics[width=1\linewidth]{Slide13.PNG}
        \caption*{The higher price causes $Q_S$ and $Q_D$ to move inward toward the autarky quantity $Q^A$.}
    \end{figure}
    }
    \only<5>{
    \begin{figure}
        \centering
        \includegraphics[width=1\linewidth]{Slide14.PNG}
        \caption*{After the tariff, the surplus triangles are formed by $P^T$, $Q_S^T$, and $Q_D^T$.}
    \end{figure}
    }
    \only<6>{
    \begin{figure}
        \centering
        \includegraphics[width=1\linewidth]{Slide15.PNG}
        \caption*{Recall: this is the free trade surplus.}
    \end{figure}
    }
    \only<7>{
    \begin{figure}
        \centering
        \includegraphics[width=1\linewidth]{Slide16.PNG}
        \caption*{The tariff causes $PS$ to increase and $CS$ to decrease.\\
        \vspace{1pt}
        Some of the lost $CS$ is converted to tax revenue, and the rest is deadweight loss.}
    \end{figure}
    }
\end{frame}

\begin{frame}{Exercise 3: Tariffs}
    Consider the same hypothetical US auto market as Exercise 2:
    \begin{itemize}
        \item $Q^S = P - \num{20000}$
        \item $Q^D = -\frac{1}{2}P + \num{40000}$
        \item $P^{RoW} = \num{30000}, \: Q_S^{FT} = \num{10000}, \: Q_D^{FT} = \num{25000}$.
        \item $CS^{FT} = \$625$ million
        \item $PS^{FT} = \$50$ million
    \end{itemize}
    \vspace{10pt}
    Suppose the government imposes a tariff which raises the price to $P^T = \num{34000}$.
    \vspace{4pt}
    \begin{enumerate}
        \item Calculate $CS^T$ and $PS^T$.
        \item Calculate tax revenue.
        \item Calculate deadweight loss.
        \item By how much does total surplus (including tax revenue) change from the free trade equilibrium?
    \end{enumerate}
\end{frame}

\begin{frame}{Exercise 3: Tariffs}
Solution:
    \begin{enumerate}
        \item $CS^T = \frac{1}{2} \times \num{23000} \times \num{46000} = \num{529000000}$.\\
        \vspace{1pt}
        $PS^T = \frac{1}{2} \times \num{14000} \times \num{14000} = \num{98000000}$.
        \vspace{2pt}
        \item $TR = (\num{23000} - \num{14000})\times(\num{34000} - \num{30000}) = \num{36000000}$.
        \vspace{2pt}
        \item $DWL = \big(\frac{1}{2}\times \num{4000} \times \num{4000} \big) + \big( \frac{1}{2} \times \num{2000} \times \num{4000} \big) = \num{12000000}$
        \vspace{2pt}
        \item $\Delta TS = 663 \text{ million} - 675 \text{ million} = -12 \text{ million}$.  (Note that this is equal to DWL).
    \end{enumerate}
\end{frame}





\end{document}

