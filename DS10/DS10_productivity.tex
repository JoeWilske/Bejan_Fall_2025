\documentclass[9pt]{beamer}
\usetheme{CambridgeUS}
\usepackage{xcolor}
\usepackage{geometry}
\usepackage{array}
\usepackage[export]{adjustbox}
\usepackage{comment}
\usepackage{caption}
\usepackage{subcaption}

\AtBeginSection[]
{
  \begin{frame}
    \frametitle{Table of Contents}
    \tableofcontents[currentsection]
  \end{frame}
}

\setbeamertemplate{footline}
{
  \leavevmode%
  \hbox{%
    \begin{beamercolorbox}[wd=.333333\paperwidth,ht=2.25ex,dp=1ex,center]{author in head/foot}%
      \usebeamerfont{author in head/foot}\insertshortauthor
    \end{beamercolorbox}%
    \begin{beamercolorbox}[wd=.333333\paperwidth,ht=2.25ex,dp=1ex,center]{title in head/foot}%
      \usebeamerfont{title in head/foot}\insertshortsubtitle
    \end{beamercolorbox}%
    \begin{beamercolorbox}[wd=.333333\paperwidth,ht=2.25ex,dp=1ex,right]{date in head/foot}%
      \usebeamerfont{date in head/foot}\insertshortdate{}\hspace*{2em}
      \usebeamertemplate{page number in head/foot}\hspace*{2ex}
    \end{beamercolorbox}
  }%
  \vskip0pt%
}

\title{Principles of Economics}
\subtitle{Discussion Session 10: Productivity}
\author{Joe Wilske}
\institute{Boston College}
\date{\today}

\begin{document}

\frame{\titlepage}

\begin{frame}{The Production Function}
    \begin{itemize}
        \item A model of the macroeconomy requires a \textbf{production function}.
        \vspace{3pt}
        \begin{itemize}
            \item Model of how different inputs $(A, L, K, H, N)$ interact to create output $Y$.
        \end{itemize}
        \vspace{3pt}
        \item Generally, 
        \begin{equation*}
            Y = A \times F(L, K, H, N)
        \end{equation*}
        where \begin{itemize}
            \item $A$ := technology
            \item $L$ := labor
            \item $K$ := capital
            \item $H$ := human capital 
            \item $N$ := natural resources
        \end{itemize}
        \vspace{3pt}
        \item A commonly used basic production function is
    \begin{equation*}
        Y = A \, L^{\alpha} \, K^{1- \alpha}, \quad \text{where} \:\: 0< \alpha < 1
    \end{equation*}
    \end{itemize}
\end{frame}

\begin{frame}{Diminishing Marginal Product}
    \begin{itemize}
        \item The production function exhibits \textbf{diminishing marginal product}:
        \vspace{5pt}
        \begin{itemize}
            \item The increase in output due to increasing an input is decreasing, holding other inputs constant.
            \vspace{5pt}
            \item $\partial F(L, K, H, N)/\partial L > 0 \:$ and $\: \partial^2 F(L,K,H,N)/\partial L^2 < 0$
        \end{itemize}
    \end{itemize}
    \begin{figure}
        \centering
        \includegraphics[width=0.45\linewidth]{production_L.png}
    \end{figure}
    \begin{itemize}
        \item[$\implies$] A capital-rich country gains more output from an increase in labor than does a labor-rich country.
    \end{itemize}
\end{frame}

\begin{frame}{Productivity}
    \begin{itemize}
        \item The ratio of capital to labor, $K/L$, is determinative of the productivity of each.
        \vspace{10pt}
        \item The ratio of output to labor, $Y/L$, tells us the productivity of labor.
        \vspace{10pt}
        \item Higher capital-to-labor ratio begets greater productivity of labor:
        \[\uparrow \frac{K}{L} \: \implies \: \uparrow \frac{Y}{L}\]
    \end{itemize}
\end{frame}

\begin{frame}{Exercise 1: Productivity}
Suppose
\vspace{4pt}
    \begin{itemize}
        \item Country A has a labor force of 50 million people and \$10 billion of capital.
        \vspace{4pt}
        \item Country B has a labor force of 25 million people and \$20 billion of capital.
    \end{itemize}
    \vspace{10pt}
    Question:  Assuming technology and other inputs are identical,
    \vspace{4pt}
    \begin{enumerate}
        \item Which country has the higher capital-to-labor ratio?
        \vspace{4pt}
        \item Which country has the higher productivity of labor?
        \vspace{4pt}
        \item Which country's GDP ($Y$) would benefit more from an influx of labor through immigration?
        \vspace{4pt}
        \item Which country's GDP would benefit more from an influx of capital through foreign investment?
    \end{enumerate}
\end{frame}

\begin{frame}{Exercise 1: Productivity}
    Solution:
    \vspace{4pt}
    \begin{enumerate}
        \item Country B.  $\: \: \frac{K_B}{L_B} = \frac{20 \times 10^9}{25 \times 10^6} = 800 > 200 = \frac{10 \times 10^9}{50 \times 10^6} = \frac{K_A}{L_A}$
        \vspace{4pt}
        \item Country B.  Higher $\frac{K}{L}$ implies higher $\frac{Y}{L}$.
        \vspace{4pt}
        \item Country B.  Higher productivity of labor implies greater benefit to GDP from additional labor.
        \vspace{4pt}
        \item Country A.  Lower $\frac{K}{L}$ implies higher productivity of capital, which implies greater benefit to GDP from additional capital.
    \end{enumerate}
\end{frame}

\begin{frame}{International Flows}
    \begin{itemize}
        \item Suppose we have a capital-rich country and a capital-poor country.
        \vspace{10pt}
        \item Given our answers on the last slide, what do we expect to happen if labor and capital are allowed to move between the two countries?
        \vspace{10pt}
        \item Does this happen in practice? (Lucas Paradox, colonial America)
    \end{itemize}
\end{frame}

\begin{frame}{Exercise 2: GDP per Capita}
Suppose
\vspace{4pt}
    \begin{itemize}
        \item Country A has a population of 240 million, and one quarter of its population is in the labor force.  Its GDP is \$12 trillion.
        \vspace{4pt}
        \item Country B has a population of 60 million, and one fifth of its population is in the labor force.  Its GDP is \$2.4 trillion.
    \end{itemize}
    \vspace{4pt}
    Question:
    \vspace{4pt}
    \begin{enumerate}
        \item Which country has the higher GDP per capita?
        \vspace{4pt}
        \item Which has the higher productivity of labor?
    \end{enumerate}
\end{frame}

\begin{frame}{Exercise 2: GDP per Capita}
    \begin{enumerate}
        \item Country A.  \[\frac{Y_A}{\text{population A}} = \frac{12 \times 10^{12}}{240 \times 10^6} = 50,000 > 40,000 = \frac{2.4 \times 10^{12}}{60 \times 10^6} = \frac{Y_B}{\text{population B}}\]
        \vspace{5pt}
        \item They are equal!  \[\frac{Y_A}{L_A} = \frac{12 \times 10^{12}}{60 \times 10^6} = 200,000 = \frac{2.4 \times 10^{12}}{12 \times 10^6} = \frac{Y_B}{L_B}\]
    \end{enumerate}
\end{frame}

\begin{frame}{Exercise 3:  Very Fun Algebra Problems}
    \begin{enumerate}
        \item If labor productivity in the United States increases by 5\% and the labor force grows by 4\%, what is its GDP growth rate?
        \vspace{10pt}
        \item If the labor force falls by 8\%, by how much must labor productivity increase for GDP growth to be positive?
    \end{enumerate}
\end{frame}

\begin{frame}{Exercise 3:  Very Fun Algebra Problems}
    \begin{enumerate}
        \item Labor productivity increase by 5\% $\implies \:$ $\frac{Y_2}{L_2}$ = $1.05 \, \frac{Y_1}{L_1}$.\\
        \vspace{5pt}
        Labor force increase by 4\% $\implies \:$ $L_2 = 1.04 \, L_1$.\\
        \vspace{5pt}
        Substitute $1.04\,L_1$ in for $L_2$ in the labor productivity growth equation and solve for $Y_2$ to find that $Y_2 = 1.092\, Y_1$.\\
        \vspace{5pt}
        $\implies \:$ GDP grows by 9.2\%.
        \vspace{10pt}
        \item Labor force falls by 8\% $\implies \:$ $L_2 = 0.92 \, L_1$.\\
        \vspace{5pt}
        Let $g$ represent the growth rate of labor productivity, so $\: \frac{Y_2}{L_2} = (1+g)\frac{Y_1}{L_1}$.\\
        \vspace{5pt}
        Substitute $0.92 \, L_1$ in for $L_2$ in the labor productivity growth equation, to get
        \[
        \frac{Y_2}{0.92\, L_1} = (1+g)\frac{Y_1}{L_1}
        \]
        Rearrange terms to find that the GDP growth rate is 
        \[
        \frac{Y_2 - Y_1}{Y_1} = 0.92(g+1) - 1
        \]
        We want to find the $g$ such that $(Y_2 - Y_1)/Y_1 > 0$, so set $0.92(g+1) - 1>0$ and solve for $g$.\\
        \vspace{5pt}
        $\implies$ $g > 8.7\%$
    \end{enumerate}
\end{frame}

































\end{document}