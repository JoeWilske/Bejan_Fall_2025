\documentclass[9pt]{beamer}
\usetheme{CambridgeUS}
\usepackage{xcolor}
\usepackage{geometry}
\usepackage{array}
\usepackage[export]{adjustbox}
\usepackage{comment}
\usepackage{caption}
\usepackage{subcaption}

\AtBeginSection[]
{
  \begin{frame}
    \frametitle{Table of Contents}
    \tableofcontents[currentsection]
  \end{frame}
}

\setbeamertemplate{footline}
{
  \leavevmode%
  \hbox{%
    \begin{beamercolorbox}[wd=.333333\paperwidth,ht=2.25ex,dp=1ex,center]{author in head/foot}%
      \usebeamerfont{author in head/foot}\insertshortauthor
    \end{beamercolorbox}%
    \begin{beamercolorbox}[wd=.333333\paperwidth,ht=2.25ex,dp=1ex,center]{title in head/foot}%
      \usebeamerfont{title in head/foot}\insertshortsubtitle
    \end{beamercolorbox}%
    \begin{beamercolorbox}[wd=.333333\paperwidth,ht=2.25ex,dp=1ex,right]{date in head/foot}%
      \usebeamerfont{date in head/foot}\insertshortdate{}\hspace*{2em}
      \usebeamertemplate{page number in head/foot}\hspace*{2ex}
    \end{beamercolorbox}
  }%
  \vskip0pt%
}

\title{Principles of Economics}
\subtitle{Discussion Session 9: The Consumer Price Index}
\author{Joe Wilske and Yuzhi Yao}
\institute{Boston College}
\date{\today}

\begin{document}

\frame{\titlepage}

\begin{frame}{GDP Deflator vs CPI}
    \begin{itemize}
        \item Prices tend to increase from year-to-year due to inflation.
        \vspace{7pt}
        \item The \textbf{GDP deflator} helps to account for inflation in GDP comparisons.
        \vspace{7pt}
        \item The \textbf{Consumer Price Index} (CPI) helps to account for inflation in cost-of-living comparisons.
        \vspace{7pt}
        \item[$\implies$] GDP includes everything produced in the US in a year, but 
        \item[$\implies$] CPI includes everything US consumers buy in a year.
        \vspace{25pt}
        \item If the price of a Honda Civic increases, it affects $\underline{\hspace{1cm}}$, but not $\underline{\hspace{1cm}}$.
        \vspace{7pt}
        \item If Caterpillar sells more excavators, if affects $\underline{\hspace{1cm}}$, but not $\underline{\hspace{1cm}}$.
    \end{itemize}
\end{frame}

\begin{frame}{Calculation of the CPI}
    \begin{itemize}
        \item As with GDP deflator, choose a base year and set to 100.
        \vspace{7pt}
        \item Unlike GDP deflator, select a ``basket of goods" to represent the typical consumer.
        \vspace{4pt}
        \begin{itemize}
            \item This will stay constant year-to-year.
        \end{itemize}
        \vspace{7pt}
        \item Set up ratios comparing the costs of the basket and the CPIs:
    \end{itemize}
    \vspace{10pt}
    \begin{equation*}
        \frac{\text{Cost base year}}{\text{Cost new year}} = \frac{\text{CPI base year}}{\text{CPI new year}}
    \end{equation*}
    Or equivalently,
    \begin{equation*}
        \frac{\text{Cost base year}}{\text{CPI base year}} = \frac{\text{Cost new year}}{\text{CPI new year}}
    \end{equation*}
    \vspace{5pt}
    \begin{itemize}
        \item From there, we can solve for `\textit{CPI new year}' or `\textit{Cost new year}'
    \end{itemize}
\end{frame}

\begin{frame}{Exercise 1: CPI \& Inflation Rate}
   The following table lists the prices of eggs and butter for the months of September, October, and November. Assume that the typical consumer buys 60 eggs and 4 sticks butter each month, and that September is the base period.
    \begin{table}[]
    \centering
    \begin{tabular}{c|c|c|c|c}
        Quantity& Total Cost & Fixed Cost & Variable Cost & Marginal Cost\\
        \hline
         0&50 & 50 & 0 & N/A\\
         1 & 150 & A & B & C\\
         2 & D & E & G & 120
    \end{tabular}
\end{table}
    \begin{enumerate}
        \item What is the consumer price index for October?
        \item What is the inflation rate for November? 
    \end{enumerate}
    \vspace{1in}
\end{frame}

\begin{frame}{Exercise 1: CPI \& Inflation Rate}
    Solution: 
    \begin{enumerate}
        \item \[\frac{\text{Cost September}}{\text{Cost October}} = \frac{\text{CPI September}}{\text{CPI October}}\]
        \vspace{5pt}
        \[\frac{134}{224.08} = \frac{100}{\text{CPI October}}\]
        \vspace{5pt}
        \[\text{CPI October} \: = \: \frac{224.08}{134} \times 100 \: \approx \: 167\]
        \vspace{7pt}
        \item Find that CPI November $\approx$ 183 in the same way, so
        \vspace{4pt}
        \[\text{Inflation Rate November} \: = \: \frac{183 - 167}{167} \times 100\% \: \approx \: 9.5\%\]
    \end{enumerate}
\end{frame}

% \begin{frame}{Exercise 2: CPI vs GDP}
%     Consider the following scenarios: 
%     \begin{itemize}
%         \item[-] The export price of soybeans produced in the US increases;
%         \item[-] The price of textile products, imported from China and bought by US consumers, increases.
%     \end{itemize}
%     What is each scenario's direct impact on the US' GDP and CPI? 
%     \vspace{1in}
% \end{frame}

% \begin{frame}{Exercise 2: CPI vs GDP}
%     Solution:
%     \begin{itemize}
%         \item[-] Scenario 1: GDP increases, CPI no change 
%         \item[-] Scenario 2: CPI increases, GDP no change
%     \end{itemize}
% \end{frame}

\begin{frame}{Exercise 2: Price Index}
    Sue Brown was an economist in 1970 and earned \$17,000 that year. Her son, Charlie Brown, is an economist today (economics is hereditary?), and he earned \$210,000 in the current year. Suppose the price index was 17.6 in 1970 and 218.4 in the current year.
    \vspace{5pt}
    \begin{enumerate}
        \item What is Sue Brown's 1970 income as converted to current-year dollars?
        \vspace{3pt}
        \item Which economist is wealthier?
    \end{enumerate}
    \vspace{2in}
\end{frame}

\begin{frame}{Exercise 2: Price Index}
    Solution:
    \begin{enumerate}
    \item
    \[
    \frac{\text{Income } 1970}{\text{Income Today}} = \frac{\text{CPI } 1970}{\text{CPI Today}}
    \] \vspace{5pt} \[
    \frac{17,000}{\text{Income Today}} = \frac{17.6}{218.4}
    \] \vspace{5pt} \[
    \text{Income Today} \: = \: \frac{218.4}{17.6} \times 17,000 \: \approx \: 210,955
    \]
    \vspace{7pt}
    \item Sue Brown was wealthier.
    \end{enumerate}
\end{frame}

\begin{frame}{Inflation and Interest}
    \begin{itemize}
        \item Interest-paying assets provide year-to-year payments on some principle.
        \vspace{5pt}
        \begin{itemize}
            \item A 1-year \$100 US Treasury Bond paid 0.4\% interest in 2022.
            \vspace{5pt}
            \item[$\implies$] A buyer receives \$100.40 at end of year.
        \end{itemize}
        \vspace{5pt}
        \item Inflation causes year-to-year devaluation of currency.
        \vspace{5pt}
        \begin{itemize}
            \item Inflation was 6.5\% in 2022.
            \vspace{5pt}
            \item[$\implies$] The value of a 2022 bond changed by $0.4\% - 6.5\% = -6.1\%$ by the end of the year!
        \end{itemize}
    \end{itemize}
    \vspace{20pt}
    \begin{itemize}
        \item \textbf{Nominal Interest Rate:} The advertised interest rate on the asset.
        \vspace{5pt}
        \item \textbf{Real Interest Rate:} The interest rate after accounting for inflation:
        \vspace{5pt}
        \[\text{Real} = \text{Nominal} - \text{Inflation}\]
    \end{itemize}
\end{frame}

\begin{frame}{Exercise 3: Inflation Rate \& Real Interest Rate}
    The following table shows real and nominal interest rates for three consecutive years:
    \begin{table}[]
    \centering
    \begin{tabular}{ccc}
    \hline
    Year & Real Interest Rate & Nominal Interest Rate  \\
    \hline
    2017 & 1\% & 2.4\% \\
    2018 & 1.5\% & 2.5\% \\
    2019 & 1.3\% & 2.1\% \\
    \hline
    \end{tabular}
\end{table}
    Supose 2017 is the base year.
    \vspace{5pt}
    \begin{enumerate}
        \item What was the inflation rate in 2018?
        \vspace{5pt}
        \item What was the CPI in 2018?
    \end{enumerate}    
\end{frame}

\begin{frame}{Exercise 3: Inflation Rate \& Real Interest Rate}
    Solution:
    \begin{enumerate}
        \item \[
        \text{Real 2018} = \text{Nominal 2018} - \text{Inflation 2018}
        \] \[
        1.5\% = 2.5\% - \text{Inflation 2018}
        \] \[
        \text{Inflation 2018} = 1\%
        \]
        \item \[
        \text{Inflation 2018} = \frac{\text{CPI 2018} - \text{CPI 2017}}{\text{CPI 2017}} \times 100\%
        \] \vspace{5pt} \[
        1\% = \frac{\text{CPI 2018} - 100}{100} \times 100\%
        \] \vspace{5pt} \[
        \text{CPI 2018} \: = \: 101
        \]
    \end{enumerate}
\end{frame}

\end{document}
