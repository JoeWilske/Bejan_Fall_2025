\documentclass[9pt, handout]{beamer}
\usetheme{CambridgeUS}
\usepackage{xcolor}
\usepackage{geometry}
\usepackage{array}
\usepackage{comment}
\usepackage[export]{adjustbox}

\AtBeginSection[]
{
  \begin{frame}
    \frametitle{Table of Contents}
    \tableofcontents[currentsection]
  \end{frame}
}

\setbeamertemplate{footline}
{
  \leavevmode%
  \hbox{%
    \begin{beamercolorbox}[wd=.333333\paperwidth,ht=2.25ex,dp=1ex,center]{author in head/foot}%
      \usebeamerfont{author in head/foot}\insertshortauthor
    \end{beamercolorbox}%
    \begin{beamercolorbox}[wd=.333333\paperwidth,ht=2.25ex,dp=1ex,center]{title in head/foot}%
      \usebeamerfont{title in head/foot}\insertshortsubtitle
    \end{beamercolorbox}%
    \begin{beamercolorbox}[wd=.333333\paperwidth,ht=2.25ex,dp=1ex,right]{date in head/foot}%
      \usebeamerfont{date in head/foot}\insertshortdate{}\hspace*{2em}
      \usebeamertemplate{page number in head/foot}\hspace*{2ex}
    \end{beamercolorbox}
  }%
  \vskip0pt%
}

\title{Principles of Economics}
\subtitle{Discussion Session 7: Consumer Choice}
\author{Joe Wilske and Yuzhi Yao}
\institute{Boston College}
\date{\today}

\begin{document}

\frame{\titlepage}

\begin{frame}{The Budget Constraint}
    Describes all combinations of $Q_A$ and $Q_B$ that are within your budget.
    \begin{figure}
        \centering
        \includegraphics[width=0.65\linewidth]{budget_region.png}
    \end{figure}
\end{frame}

\begin{frame}{Exercise 1: Budget Constraint}
    Q1: Consider two goods, A and B, with $P_A=\$1$ and $P_B=\$2$. Suppose income is \$100. 
    \vspace{7pt}
    \begin{enumerate}
        \item Draw the budget constraint and label the intercepts.
        \vspace{7pt}
        \item How will the budget constraint be affected if 
        \vspace{5pt}
        \begin{itemize}
            \item[-] $P_A$ increases to \$4?
            \vspace{5pt}
            \item[-] Income increases to \$120?
        \end{itemize}
    \end{enumerate}
    \vspace{1.5in}
\end{frame}

\begin{frame}{Exercise 1: Budget Constraint}
    Solution: 
    \begin{figure}
        \centering
        \includegraphics[width=0.9\linewidth]{sol1.png}
    \end{figure}
\end{frame}

\begin{frame}{Utility Function}
\begin{itemize}
    \item Describes the satisfaction or ``utility" that one receives from consuming a good
    \item Concave $\implies$ diminishing marginal utility
\end{itemize}
    \begin{figure}
        \centering
        \includegraphics[width=0.5\linewidth]{utility_function.png}
    \end{figure}
    \begin{itemize}
        \item But what if we want to compare the utility from both goods $A$ and $B$?
    \end{itemize}
\end{frame}

\begin{frame}{Utility in 3 Dimensions}
\begin{columns}[c]
\begin{column}{0.5\textwidth}
\begin{itemize}
    \item Add another axis!\\
    $\implies$ Utility depends on $Q_A$ and $Q_B$
    \vspace{5pt}
    \item Can translate this to 2D by depicting curves where utility is constant
    \vspace{5pt}
    \begin{itemize}
        \item Like a topographical map
    \end{itemize}
\end{itemize}
\end{column}
\begin{column}{0.5\textwidth}
    \includegraphics[width=\linewidth]{utility1.png}
    \end{column}
\end{columns}
\end{frame}

\begin{frame}{Indifference Curves}
    \begin{itemize}
        \item Every combination of $A$ and $B$ along the same curve gives the same utility
        \item The consumer is ``indifferent" between all points on the same curve
    \end{itemize}
    \begin{figure}
        \centering
        \includegraphics[width=0.5\linewidth]{indifference_curves.png}
    \end{figure}
\end{frame}

\begin{frame}{Marginal Rate of Substitution}
    \begin{itemize}
        \item MRS:  How much $B$ I'm willing to give up for one unit of $A$.
        \item Just the negative slope of the indifference curve between two points:\\$MRS = -(B_1 - B_2)/(A_1 - A_2)$
    \end{itemize}
    \centering
    \includegraphics[width=0.5\linewidth]{mrs.png}
    \begin{itemize}
        \item In this case, $MRS = 3/4$.  ``I can give up 3/4 bananas for 1 apple and be indifferent".
    \end{itemize}
\end{frame}

\begin{frame}{Optimal Consumption Point}
    \begin{itemize}
        \item Consumer maximizes utility subject to budget constraint.
        \item ``Climb as high up the hill as possible" while staying behind the BC.
    \end{itemize}
    \begin{figure}
        \centering
        \includegraphics[width=0.5\linewidth]{optimum.png}
    \end{figure}
\end{frame}

\begin{frame}{Optimal Consumption Point}
    \begin{itemize}
        \item Consumer maximizes utility subject to budget constraint.
        \item ``Climb as high up the hill as possible" while staying behind the BC.
    \end{itemize}
    \begin{figure}
        \centering
        \includegraphics[width=0.5\linewidth]{optimum.png}
    \end{figure}
    \begin{itemize}
        \item<2-> At point $C$, the (negative) slope of the BC is equal to the MRS (slope of the IC)
    \end{itemize}
\end{frame}

\begin{frame}{Exercise 2: Indifference Curve \& Optimal Choice}
    Q1: Consider the following indifference curve.
    \begin{enumerate}
        \item Calculate the marginal rate of substitution (MRS) from point B to C
        \item Suppose the price of ice cream bars and strudels are both \$10. Your income is \$110. Draw and label your budget constraint and mark the optimal choice in relation to an indifference curve. 
        \item Suppose your income increases to \$220. How will your optimal choice change if 
        \begin{itemize}
            \item[-] ice cream bar is a normal good?
            \item[-] ice cream bar is an inferior good? 
        \end{itemize}
        How must your indifference curves be shaped in each of those scenarios?
    \end{enumerate}
    \begin{figure}
        \centering
        \includegraphics[width=0.4\linewidth]{fig1.jpg}
    \end{figure}
\end{frame}

\begin{frame}{Exercise 2: Indifference Curve \& Optimal Choice}
    Solution: 
    \begin{enumerate}
        \item $MRS=\frac{34-23}{11-7}=\frac{11}{4}$
        \item The optimal choice is the tangent point of the indifference curve to the budget line 
        \begin{figure}
            \centering
            \includegraphics[width=0.35\linewidth]{sol2_a.png}
        \end{figure}
        \item Changes in optimal choice: 
        \begin{figure}[h!]
            \centering
            \includegraphics[width=0.8\linewidth]{sol2_b.png}
        \end{figure}
    \end{enumerate}
\end{frame}

\begin{frame}{Externalities}
    \begin{itemize}
        \item When a market transaction effects a third party.
        \vspace{5pt}
        \item \textbf{Negative Externality:} The third party is negatively effected.
        \vspace{3pt}
        \begin{itemize}
            \item Quantity produced is too high.
            \item Policy response is a tax.
        \end{itemize}
        \vspace{5pt}
        \item \textbf{Positive Externality:} The third party is positively effected.
        \vspace{3pt}
        \begin{itemize}
            \item Quantity produced is too low.
            \item Policy response is a subsidy.
        \end{itemize}
    \end{itemize}
\end{frame}

\begin{frame}{Exercise 3: Externalities}
    Q1: Consider the following market for beers:
    \begin{itemize}
        \item[-] $Q^D = 20 - P$
        \item[-] $Q^S = P - 2$
    \end{itemize}
    Suppose that drinking beers at a party makes the party more fun for everybody, including non-drinkers, to the tune of \$2 of fun per beer.
    \vspace{5pt}
    \begin{enumerate}
        \item Find the market equilibrium quantity.
        \item What kind of externality is this?
        \item What is the appropriate policy response by non-drinkers?
        \item How big should the response be? 
        \item Assume the policy is implemented. Find the socially optimal quantity.
    \end{enumerate}
    \vspace{1in}
\end{frame}

\begin{frame}{Exercise 3: Externalities}
    Solution:
    \begin{enumerate}
        \item $P=11, \: Q = 9$
        \item Positive
        \item Non-drinkers should subsidize beers for drinkers
        \item The size of the externality: $\sigma = \$2$
        \item $P^D = P^S - 2 \:$, so 
    \end{enumerate}
    \begin{align*}
        Q^D(P^D) &= Q^S(P^S)\\
        20 - P^D &= P^S - 2\\
        20 - (P^S - 2) &= P^S - 2\\
        \implies \: P^S &= 12,\\
        P^D &= 10,\\
        Q^{\textrm{optimal}} &= 10.
    \end{align*}
\end{frame}


\end{document}
